%--------------------------------------------------------------------------------
% Text příspěvku do sborníku EEICT
%
% Vytvořil:  Petr Muller
% Datum:     3.3.2010
% E-mail:    xmulle13@stud.fit.vutbr.cz
%
%--------------------------------------------------------------------------------
%
\documentclass{eeict}
\inputencoding{utf8}
\usepackage[bf]{caption2}
%--------------------------------------------------------------------------------


\title{A~TOOL FOR ANALYSIS OF DYNAMIC MEMORY ALLOCATORS}
\author{Petr Muller}
\programme{Master Degree Programme (2), FIT BUT}
\emails{xmulle13@stud.fit.vutbr.cz}

\supervisor{Tomáš Vojnar}
\emailv{vojnar@fit.vutbr.cz}

\begin{document}
% -- Hlavička práce --

\maketitle

% -- Abstrakt práce --
\selectlanguage{english}
\abstract{This paper describes a tool for the observation and analysis of the most important dynamic memory allocator metrics under various use cases. At first the problem which this tool aims to solve is introduced, following with the tool design description, which is the main focus of the paper. Current status of the work is briefly discussed in the conclusion.}
%-------------------------------------------------------------------------------
\selectlanguage{english}
\section{Introduction}

Dynamic memory allocators (abbreviated DMA) are fundamental part of the operating systems, providing memory allocated at runtime to the programs. This is something almost every program needs; it is not neither possible nor effective to allocate the whole needed memory at the compile time. When a program is doing lot of memory allocation requests, its performance partially depends on the allocator's performance. Performance does not have a clear meaning here: it can mean either the temporal (how quickly the program performs it's tasks) or spatial (program is using the minimal space needed) effectivity. This DMA performance is affected by the usage pattern, which is determined by the user program. Operating system vendors usually provide a generic DMA tuned to perform "well" for the most usual use cases and standard programs. But sometimes the performance is crucial and using a different DMA performing exceptionally well for one specific program or in one specific environment can improve it.

To be able to decide about using specific DMA, some amount of information about their advantages and disadvantages is needed. The tool aims to provide all the relevant information about DMA performance for arbitrary usage pattern.

\section{Performance problems caused by poor DMA choice}

The introduction mentioned two DMA performance areas: spatial and temporal effectivity. Both of these criteria are affected by the method and algorithm which the DMA uses to provide memory chunks to the programs. DMA usually requests large areas of memory from the operating system. In Linux, this memory is virtual, so the space appears continuous to both the allocator and the user program, even when it is not continuous physically~\cite{drepper}. DMA operates on this continuous space, providing smaller parts of it to the user program. The dynamic allocation of storage is an online algorithm, which means the DMA must respond to the request immediatelly without knowledge about the future requests, and it cannot change its decision later. This leads to the fact that a choice made can later turn out to be suboptimal, or even totally wrong. It can be proven that a DMA optimal for all usage patters cannot be constructed~\cite{dsa}.

The spatial effectivity of DMA is quite straightforward --- the only criteria is that a DMA should not waste space already provided by the operating system. Such a wasted space is called {\em fragmentation}. Two kinds of fragmentation exist: external and internal. Internal fragmentation is the space wasted directly by the DMA decision: alignment gaps, metadata and another chunks of memory not to be provided to the user program. Second type of fragmentation, external, is a phenomenon caused by the poor memory chunk placement. External fragmentation is technically free storage which could be provided to the user program, but it cannot because it is too small and the program requests larger chunk. Both types of fragmentation can cause a large amount of unnecessary memory consumption. In worst case, the program running for a long time can even exhaust all the memory the OS can provide, even when it does not contain poor memory allocation code.

The temporal effectivity is a bit more complicated. There are two types in this area. The first is straightforward: the cost of one (de)allocation request. If this cost is high, program doing lot of allocation requests can have poor performance. The cost is directly determined by the algorithm used in the DMA, with additional factors like the cost of appropriate system calls. The second type is the performance of the program as a whole. In certain environments, the memory placement itself can affect the performance of the program. The example of such environment can be the multithreaded program running on a multiprocessor computer with both per-cpu and shared caches~\cite{drepper}. Poor placement of memory chunks used by different threads can lead to problems such as false sharing or cache trashing. In threaded environment, performance can be hit also by the DMA design; if the allocator is considered a critical section and is guarded by a lock, threads cannot allocate memory simultaneously and have to wait for each other. The DMA can be designed to deal with this situation by having several "arenas" for different threads. This introduces an unnecessary overhead for the programs using a single thread in exchange.

Stock DMAs usually perform well with most use cases. But for certain programs, they do not perform well enough. Different DMA can improve the performance. To allow a choice, the tool has to provide information how the DMA suffers by the described problems.

\section{Design of the analysis tool}

In order for the analysis tool to be universal, it needs to be able to provide measurements and other data on as wide area of program use cases as possible. Very important is the ability to provide information about DMA performance when it is used by a specific program, so the creators of such program can determine a best DMA when performance is crucial.

The tool consists of two parts: one part creates a scenario program with the memory allocator usage pattern, the second part runs the scenario and collects data. The scenario is described in a domain specific language and it carries the information about threads and the order, size and length (how long the program keeps the allocated memory before returning it) of the allocations done by different threads. This file can either be created by the user, generated randomly, or created by the tool capturing the usage pattern of a real running program. Several approaches were considered for the capturing tool, with the probable outcome of combination of using LD\_LIBRARY\_PATH~\cite{cman} with fake malloc implementation and of capturing malloc and free calls with external tool such as ltrace or Systemtap. These approaches have their advantages and disadvantages, which are not discussed here due to the space constraints.

This scenario file is translated into the standard C program. In addition to the memory allocation and freeing commands, this program contains hooks where the data collection probes can register to get the data about the execution. It is then compiled to the binary by using standard C compiler such as GCC. In order to test the same scenario binary with different DMA implementations, the memory allocator will be linked to the binary either at runtime (using dlopen~\cite{cman} interface), or using the LD\_LIBRARY\_PATH~\cite{cman} environment variable.

The scenario binary is run multiple times, enough to provide sufficient amount of samples to obtain statistically sound result. Different metrics are collected by separate programs either hooked internally in the binary, or external tools observing the behavior of the program. Different metrics need different approaches to collection. For example, the probes which collect the data about the spatial memory layout in the process' address space after each memory allocation request take some time. If these probes were running simultaneously with the probe measuring the total scenario runtime, the time measurement would be incorrect, because they would contain the overhead of the layout examination. To avoid the incorrect results, the harness lets the user choose the metrics he wants to collect, and warns about the conflicting ones. Only the probes for the selected metrics will be collecting data in one scenario run.

After the run, the collected data will be analysed and statistical output provided to the user. For certain metrics, a visual information can be also provided, such as graphs showing various metrics projected to runtime. Another useful output is the visualization of the process' address space, showing free memory and memory owned by different threads. This can be extended with a "player" showing this visualization in time. For all metrics, the tool will provide the most important information about the impact on performance of the particular scenario.

\section{Conclusion}

This paper described a design of a tool providing information about various aspects of DMA performance. This design is based on a study of the possible problems caused by DMA, and identification of the factors in both environment and the usage pattern affecting performance. In the current state of the project, several prototypes of various data collecting tools are created, and they will be implemented in the compact harness. The draft versions of the scenario generator, simulator and program usage pattern capturing tools were made and tested.

\begin{thebibliography}{9}
   \bibitem{drepper} Drepper, U.: What every programmer should know about memory, 2007,
                     <http://people.redhat.com/drepper/cpumemory.pdf> [online]
   \bibitem{dsa} Wilson, Paul R., Johnstone, Mark S., Neely, M., Boles, D.: Dynamic Memory Allocation: A~survey and critical review. In: Proc. Int. Workshop on Memory Management, Kinros, Scotland, 1995
   \bibitem{cman} Loosemore, Sandra et al.: The GNU C Library Reference Manual, 2007, Free Software Foundation

\end{thebibliography}

\end{document}
